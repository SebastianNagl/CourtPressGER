% This is samplepaper.tex, a sample chapter demonstrating the
% LLNCS macro package for Springer Computer Science proceedings;
% Version 2.21 of 2022/01/12
%
\documentclass[runningheads]{llncs}
%
\usepackage[T1]{fontenc}
% T1 fonts will be used to generate the final print and online PDFs,
% so please use T1 fonts in your manuscript whenever possible.
% Other font encondings may result in incorrect characters.
%
\usepackage{graphicx}
% Used for displaying a sample figure. If possible, figure files should
% be included in EPS format.
%
% If you use the hyperref package, please uncomment the following two lines
% to display URLs in blue roman font according to Springer's eBook style:
\usepackage{hyperref}
\usepackage{color}
\renewcommand\UrlFont{\color{blue}\rmfamily}
\urlstyle{rm}
%
\begin{document}
%
\title{CourtPressGER: Generating German Court Press Releases with LLMs}
%
\titlerunning{CourtPressGER}
% If the paper title is too long for the running head, you can set
% an abbreviated paper title here
%
\author{Sebastian Nagl\inst{1}\orcidID{0000-0000-0000-0000} \and
Mohamed Elganainy\inst{1}\orcidID{0000-0000-0000-0000} \and
Matthias Grabmair\inst{1}\orcidID{0000-0000-0000-0000}}
%
\authorrunning{Nagl et al.}
% First names are abbreviated in the running head.
% If there are more than two authors, 'et al.' is used.
%
\institute{Technical University of Munich, Munich, Germany\\
\email{\{sebastian.nagl,mohamed.elganainy,matthias.grabmair\}@tum.de}}
%
\maketitle              % typeset the header of the contribution
%
\begin{abstract}
This paper presents CourtPressGER, a system for automatically generating German court press releases using Large Language Models (LLMs). We present a curated dataset with 6.4k entries of court decisions with corresponding press releases from Germany's highest courts. The dataset is enhanced with synthetic prompts that enable automated generation of press releases from court decisions. We describe a pipeline for generating press releases with various state-of-the-art models and evaluate the results using human and automated metrics. Our approach combines specialized legal language models with domain-specific techniques to produce accurate and informative press releases that adhere to journalistic and legal standards.

\keywords{Legal NLP \and Legal AI \and Court Press Releases \and German Legal Text Generation \and Large Language Models}
\end{abstract}
%
%
%
\section{Introduction}
The German legal system consists of a complex network of courts that regularly publish extensive decisions. To make these decisions accessible to the public, the highest courts create press releases that summarize the essential aspects and implications of the decisions in an understandable form. These press releases serve as an important interface between the judicial system and the general public by explaining complex legal matters in an accessible way.

However, the manual creation of such press releases requires significant resources. At the same time, recent advances in Large Language Models (LLMs) offer new possibilities for automated text generation in specialized domains. Our project CourtPressGER aims to leverage these capabilities for the automatic generation of court press releases.

\subsection{Objectives}
The main objectives of our project are:
\begin{itemize}
    \item The creation of a curated dataset with 6.4k entries of court decisions with corresponding press releases from Germany's highest courts.
    \item The development of synthetic prompts for each decision-press release pair that can be used to automatically generate press releases.
    \item The implementation of a pipeline that queries various LLMs with the synthetic prompts and stores the results alongside the actual press releases.
    \item The evaluation of the generated press releases using human and automated metrics.
\end{itemize}

\section{Related Work}
\subsection{Juristic Text Generation}
In recent years, the automatic generation of juristic texts has made significant progress with the emergence of powerful language models. Various studies have focused on summarizing juristic documents, generating juristic arguments, and simplifying complex legal texts.

\subsection{Large Language Models in the Juristic Context}
LLMs have increasingly been used for juristic applications, including legal advice, document analysis, and decision predictions. Current research explores the ability of LLMs to understand and generate legal language, as well as their reliability and ethical implications in the legal context.

\subsection{Press Releases in the Judicial System}
Studies on the role and impact of court press releases have highlighted their importance for public perception and understanding of legal decisions. However, there has been little research on the automated generation of such releases, especially in the German legal system.

\section{Dataset}
\subsection{Data Sources}
Our dataset includes court decisions and corresponding press releases from Germany's highest courts, including the Federal Constitutional Court, the Federal Court of Justice, and other federal courts. The data was collected from publicly accessible sources.

\subsection{Data Cleaning}
The data cleaning process involved a combination of rule-based methods and semantic similarity analysis. The process included:
\begin{itemize}
    \item Filtering and normalizing raw data
    \item Validating the mappings between decisions and press releases
    \item Removing duplicates and inconsistent entries
\end{itemize}

\subsection{Dataset Statistics}
The cleaned dataset contains 6.4k pairs of court decisions and press releases. The average length of decisions is X Tokens, while press releases average Y Tokens in length. The dataset covers various legal fields, including civil, criminal, administrative, and constitutional law.

\section{Methodology}
\subsection{Synthetic Prompts}
For each decision-press release pair, we developed synthetic prompts that serve as input for LLMs to generate press releases. These prompts were carefully designed to highlight the key aspects of the decision and to train the models to create relevant and precise press releases.

\subsection{Pipeline for Generating Press Releases}
Our pipeline includes various LLMs, which can be categorized into three groups:
\begin{itemize}
    \item Large Models: GPT-4o, Llama-3-70B
    \item Small Models: Teuken-7B, Llama-3-8B, EuroLLM-9B
\end{itemize}

The pipeline is designed to send the synthetic prompts to the models, collect the generated press releases, and store them alongside the actual press releases. A checkpoint system allows for the continuation of interrupted generation processes.

\subsection{Evaluation Metrics}
We use various metrics to evaluate the generated press releases:
\begin{itemize}
    \item ROUGE (Rouge-1, Rouge-2, Rouge-L)
    \item BLEU (BLEU-1 to BLEU-4)
    \item METEOR
    \item BERTScore (with EuroBERT model)
\end{itemize}

In addition to these automated metrics, we conduct a human evaluation where experts assess the quality, accuracy, and understandability of the generated press releases.

\section{Results}
\subsection{Comparison of Models}
[Here follow detailed results of the various models based on automated metrics.]

\subsection{Human Evaluation}
[Here follow the results of the human evaluation, including qualitative observations and quantitative assessments.]

\subsection{Analysis of Case Studies}
[Here, selected examples are analyzed that highlight specific strengths or weaknesses of the generated press releases.]

\section{Discussion}
\subsection{Interpretation of Results}
[Here follows a discussion of the main findings and their implications.]

\subsection{Strengths and Weaknesses}
[Here, we discuss the strengths and weaknesses of our approach, including specific challenges in generating legal texts in German.]

\subsection{Ethical Considerations}
[Here, we discuss ethical issues in the context of automated generation of court press releases, including transparency, accountability, and potential biases.]

\section{Conclusion and Future Directions}
Our project CourtPressGER demonstrates the potential of LLMs for automated generation of court press releases in the German legal system. The results show that modern language models are capable of producing understandable and informative press releases that accurately represent complex legal matters.

Future research directions include the integration of domain-specific knowledge bases, the improvement of accuracy and reliability through fine-tuning of the models, and the extension of the approach to other languages and legal systems.

\begin{credits}
\subsubsection{\ackname} This research was supported by [Funder/Project Number, if applicable]. This paper will be presented at the ICAIL Workshop 2025.

\subsubsection{\discintname}
The authors declare that they have no conflicts of interest that are relevant to the content of this paper.
\end{credits}
%
% ---- Bibliography ----
%
% BibTeX users should specify bibliography style 'splncs04'.
% References will then be sorted and formatted in the correct style.
%
% \bibliographystyle{splncs04}
% \bibliography{mybibliography}
%
\begin{thebibliography}{8}
\bibitem{ref_article1}
Author, A.: Article Title. Journal \textbf{2}(5), 99--110 (2016)

\bibitem{ref_lncs1}
Author, B., Author, C.: Title of a Conference Paper. In: Editor, D., Editor, E. (eds.) CONFERENCE 2016, LNCS, vol. 9999, pp. 1--13. Springer, Heidelberg (2016). \doi{10.10007/1234567890}

\bibitem{ref_book1}
Author, F., Author, G., Author, H.: Book Title. 2nd edn. Publisher, Location (1999)

\bibitem{ref_proc1}
Author, I.: Contribution Title. In: 9th International Conference, pp. 1--2. Publisher, Location (2010)

\bibitem{ref_url1}
German Federal Court Website, \url{http://www.example.de}, last accessed 2024/06/10
\end{thebibliography}
\end{document}
